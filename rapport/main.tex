\documentclass{article}

\usepackage{geometry}
 \geometry{
 a4paper,
 total={170mm,257mm},
 left=25mm,
 right=25mm,
 top=25mm,
 bottom=25mm,
 }

 % A little black magic
\usepackage{microtype}

\usepackage{stmaryrd}
\usepackage{biblatex}
\usepackage[T1]{fontenc}
\usepackage{enumitem}
\usepackage{graphicx}
\usepackage{caption}
\usepackage{subcaption}
\usepackage{amsthm}
\usepackage{amssymb}
\usepackage[french]{babel}
% \usepackage[autolanguage]{numprint} % for the \nombre command
\usepackage{hyphenat}
% \usepackage{minted}
% \setminted[ocaml]{style=vs}

% \addbibresource{bibliographie.bib}

% \captionsetup{labelformat=empty}

% \newtheorem{definition}{Définition}[section]
% \newtheorem{proposition}[definition]{Proposition}
% \newtheorem{conjecture}[definition]{Conjecture}
% \newtheorem{exemple}[definition]{Exemple}

\title{Construction des espaces de Eilenberg-MacLane en HoTT}

\author{Stage L3 de Camil Champin}
\author{Camil Champin\\[1ex]
\small Encadré par Emile Oleon et Samuel Mimram}  % use a smaller font size
\date{Juin-Juillet 2023}

\begin{document}

\maketitle

\section{Introduction du problème}

% Ceci devrait s'écrire tout seul

\section{Construction de $K(G,1)$ par torsions}

\subsection{Construction formelle}
\subsection{Correction}
\subsection{Problème}

\section{Construction concise à partir de générateurs}

\subsection{Exemple des groupes cycliques}
\subsection{Généralisation}
\subsection{Exemple sur un autre groupe} % groupe dihédral ?
\subsection{Correction}

\end{document}
