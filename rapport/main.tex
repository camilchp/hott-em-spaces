\documentclass{article}

% Eshtetic packages -----------------------
\usepackage{geometry}
 \geometry{
 a4paper,
 total={170mm,257mm},
 left=25mm,
 right=25mm,
 top=25mm,
 bottom=25mm,
 }

 % A little black magic
\usepackage{microtype}

% Other Packages --------------------------

\usepackage{stmaryrd}
\usepackage{biblatex}
\usepackage[T1]{fontenc}
\usepackage{enumitem}

\usepackage{graphicx}
\graphicspath{ {../images/} }

\usepackage{multicol}

\usepackage{caption}
\usepackage{subcaption}
\usepackage{amsthm}
\usepackage{amsmath}
\usepackage{amssymb}
\usepackage{bbold}
\usepackage[french]{babel}
\usepackage{csquotes}
% \usepackage[autolanguage]{numprint} % for the \nombre command
\usepackage{hyphenat}
% \usepackage{minted}
% \setminted[ocaml]{style=vs}

\usepackage{tikz}
%\usepackage{tikz-cd}
\usetikzlibrary[cd]
\usetikzlibrary{matrix, arrows}

% Operators -------------------------------

\let\loop\relax
\DeclareMathOperator{\ua}{ua}
\DeclareMathOperator{\loop}{loop}
\DeclareMathOperator{\transp}{transp}
\DeclareMathOperator{\inv}{inv}
\DeclareMathOperator{\id}{id}
\DeclareMathOperator{\isconnected}{isConnected}
\DeclareMathOperator{\gsetstr}{GSetStr}
\DeclareMathOperator{\isprop}{isProp}
\DeclareMathOperator{\isset}{isSet}
\DeclareMathOperator{\isgroupoid}{isGroupoïd}
\DeclareMathOperator{\isconn}{isConn}
\DeclareMathOperator{\isequiv}{isEquiv}
\DeclareMathOperator{\pathtoequiv}{pathToEquiv}
\let\Pr\relax
\DeclareMathOperator{\Pr}{Pr}
\DeclareMathOperator{\shape}{shape}
\DeclareMathOperator{\base}{base}
\DeclareMathOperator{\gset}{G-Set}
\DeclareMathOperator{\torsor}{Torsor}
\DeclareMathOperator{\aut}{Aut}
\DeclareMathOperator{\code}{code}
\DeclareMathOperator{\decode}{decode}
\DeclareMathOperator{\transport}{transport}
\DeclareMathOperator{\refl}{refl}
\DeclareMathOperator{\baut}{BAut}
\DeclareMathOperator{\set}{Set}
\DeclareMathOperator{\prop}{Prop}
\DeclareMathOperator{\groupa}{Group}
\let\hom\relax
\DeclareMathOperator{\hom}{Hom}

% \addbibresource{bibliographie.bib}

% \captionsetup{labelformat=empty}

%\renewcommand{\labelitemi}{$\bullet$}

\newtheorem{definition}{Définition}[section]
\newtheorem{theorem}{Théorème}[section]
\newtheorem{axiom}{Axiom}[section]
\newtheorem{proposition}[definition]{Proposition}
\newtheorem{lemme}[definition]{Lemme}
% \newtheorem{conjecture}[definition]{Conjecture}
% \newtheorem{exemple}[definition]{Exemple}


\title{A minimal construction of Eilenberg-MacLane spaces in HoTT}

\begin{document}

\tableofcontents

\section{Introduction to the Problem : Eilenberg-MacLane Spaces}

\subsection{Reminders on Topology}

Let $A$ be a topological space. We call continuous functions from $[0;1]$ to $A$ the \emph{paths of $A$}. If $x,y$ are points of $A$, we say that $\gamma : [0;1] \to A$ is a path from $x$ to $y$ if $\gamma(0) = x$ and $\gamma(1) = y$. Paths from $x$ to $x$ are called the \emph{loops} of $x$. A homotopy between two paths $p$ and $q$ is a path between paths, which one can think of as a continuous family of paths $F_{t} : [0;1] \to [0;1] \to A$ such that $F_{0} = p$ and $F_{1} = q$.

INSERT DRAWING

\subsection{The Fundamental Group}

Let us consider the loops of a point $x$ in $A$. We can concatenate paths (we need only to stretch the input interval back to $[0;1]$), and take the inverse of any path (the path that goes backwards through the same points). It is a fundamental result in algebraic topology that these paths considered up to homotopy form a group. Furthermore if $A$ is path connected, the group is isomorphic for any choice of $x$ in $A$. We call this group (or rather family of groups) the \emph{fundemental group} of $A$, which we denote $\pi_{1}(A)$.

One can also define $\pi_{2}(A)$ to be the group of homotopies between loops of any $x$ in $A$, up to homotopies of homotopies (This is the fundamental group of the space of paths from $x$ to $x$) and so on for any $\pi_{n}$. ($\pi_{0}(A)$ is the set of connected components of $A$).

\subsection{Eilenberg-MacLane Spaces}

It is a non trivial fact that for any group $G$, there exists a topological space X such that :

\[
\pi_{n}(X) =
\begin{cases}
  G \text{ if } n = 1 \\
  \mathbb{1} \text{ if } n \neq 1
\end{cases}
\]

The point of this internship was to study how such a space can be constructed using the formalism of Homotopy Type Theory, and to show that a minimal construction can be achieved when given generators of the group $G$.

\section{Introduction to Homotopy Type Theory}

\subsection{What is Homotopy Type Theory?}

Types can be thought of as set-like collections, with inhabitants $x : A$ being seen as `elements' of a type. Types can be broader than sets though, one can write the type of all groups for example. To avoid Russel's paradox, we consider every type to be an inhabitant of a broader type called a universe, often denoted $\mathcal{U}$, rather than allowing a type to inhabit itself. We know from the Curry-Howard correspondance that we can also see them as propositions, with inhabitants being seen as proofs. Functions for example are proofs of implications $A \to B$, in the sense that they map proofs of $A$ to proofs of $B$. Homotopy type theory sees types as topological spaces, and inhabitants as points of that space.


\subsection{Reminders on Dependant Types}

Dependant type theory allows us to express formulas of first-order logic as types, namely it is needed for $\exists$-statements and $\forall$-statements. If $P : A \to \mathcal{U}$, ($P$ is called a `type family' over $A$), we have types:

\[\sum_{x : A}P(x) \; \; ; \; \; \prod_{x : A}P(x) \]

Called a dependant sum and dependant product. Their inhabitants are respectively: proofs of an $\exists$-stament, i.e.\ dependant pairs $(x , p)$ where $x : A$ and $p : P(x)$, and proofs of a $\forall$-statement, i.e dependant functions that map every inhabitant $x : A$ to an inhabitant of $P(x)$.

\subsection{Equality Types}

As opposed to classical logic, in type theory equality is part of the syntax. Therefor there are two notions of equality: let $x,y : A$. There is \emph{judgemental equality}, (noted $x :\equiv y$) which is used to define things, and \emph{propositional equality} (noted $x =_{A} y$ or just $x = y$) which is the kind that we proove. Judgemental equality is above the syntax: $x :\equiv y$ just means we allow ourselves to swap out the symbols at any point. However, $(x = y)$ is a type (a so called `identity type'). The type $(x = x)$ is always inhabited, namely by an inhabitant called \emph{reflexivity}: $\refl : (x = x)$. Furthemore, identity types of $A$ with $x$ on the right hand side have an induction principal $J$ called \emph{path induction}:

\[J : \prod_{P : \prod_{y : A} (x = y) \to \mathcal{U}} P(x, \refl) \to \prod_{y: A} \prod_{p : (x = y)} P(y,p)\]

Wich reads `If $P$ is a postulate on $y$ equal to $x$, and $P(x,refl)$ holds, then for any $y$ equal to $x$ over any path $p$, $P(x,y)$ holds.'

We note that if $p,q : (x = y)$, there is also a type $(p = q)$ which may or may not be inhabited. In fact there is an infinite structure of equalities between paths. This is where Homotopy Type Theory (HoTT) comes in: $x$ and $y$ are points of $A$, $p$ and $q$ are paths in that space from $x$ to $y$ (think continuous function $\left[ 0;1\right] \to A$). Inhabitants of $(p = q)$ are homotopies between these paths etc\ldots Reflexivity is the constant path, but there may be different paths that are not homotopic. Points in the path may be thought of as inhabitants of that space.

INSERT DRAWING

Following this interpretation, the type of equalities of a point $a$ in a type $A$ is perfectly analogous to the loop-space of $a$ in the space $A$. Therefor we define the loop operator, which maps a \emph{pointed type} (a type and an inhabitant) to its loop space.

\[\Omega(A , a_0) :\equiv (a_0 = a_0)\]

\subsection{Higher Inductive Types}

Just like Martin-Löf type theory, HoTT allows us to define types inductively (like the natural numbers or lists), but it also allows us to define what are called \emph{Higher Inductive Types} (HITs). These are types in which both inhabitants and paths (including higher paths) are inductively defined. To understand this report, all that is needed is the following intuition: a HIT is an inductive type in which the higher structure is explicitly defined. The most common example is the circle $\mathcal{S}^{1}$: it is defined with a point ($\base : \mathcal{S}^{1}$) and a path ($\loop : \base = \base$).

INSERT DRAWING

One can show that $\loop$ is not equal to $\refl$, or to $\loop \cdot \loop$, etc\ldots This way, one shows that $\pi_{1}(\mathcal{S}^{1}) = \mathbb{Z}$.

\subsection{Type Levels}

We've just seen how two inhabitants of $(x = y)$ may be equal in several different ways, but we may like to work with types in which all equalities are unique. Let us generalize and formalize this idea:

\begin{definition}
  We say a type $A$ is a \emph{Proposition} if all of it's inhabitants are equal.
  \[\isprop(A) :\equiv \prod_{x,y : A}(x = y)\]
\end{definition}

\begin{definition}
  We say that a type is a \emph{Set} if all of it's identity types are propositions. We say that a type is a \emph{Groupoïd} if all of it's identity types are sets. In general we call an \emph{$n$-type} any type who's identity types at level $n$ are propositions (A proposition is called an $(-1)$-type).
\end{definition}

Note that a proposition is always a set, which is always a groupoïd, etc\ldots. In fact, seeing this allows one to show that $\isprop(A)$ is itself a proposition.

INSERT DRAWING

\subsection{Equivalences and  Univalence Axiom}

\begin{definition}
  A function $f : A \to B$ is said to be an \emph{equivalence} if it has an left and a right inverse:
  \[\isequiv(f) :\equiv \left(  \sum_{g : B \to A} f \circ g = \id_{B} \right) \times \left(  \sum_{h : B \to A}h \circ f = \id_{A} \right) \]
\end{definition}

It is important to note that most natural definitions of an equivalence are logically equivalent to this one, however this is the simplest definition that is propositionnal. We now come to one of the most important points of Homotopy Type Theory:

\begin{definition}
  We say that two types are \emph{equivalent} if there is an equivalence between them:
  \[(A \simeq B) :\equiv \sum_{f : A \to B} \isequiv(f) \]
\end{definition}

The identity function on a given type is obviously an equivalence, therefor by path induction one can define a very natural map:

\[\pathtoequiv : (A =_{\mathcal{U}} B) \to (A \simeq B)\]

\begin{axiom}[Univalence]
  The function $\pathtoequiv$ is an equivalence. Its inverse is $\ua$. Namely:
  \[(A = B) \simeq (A \simeq B)\]
\end{axiom}

This axiom, introduced by Voevodsky in 2006, is one of the key features of Homotopy Type Theory. One powerful application of this axiom is that if two algebraic structures are isomorphic, they can be identified. In classical logic mathematicians often consider these structures `up to isomorphism' and identify them seamlessly, univalence allows this formally. Let us study an example I formalized during my internship with the Agda proof-assistant: the structure of a G-set.

\subsection{Transport}\label{transport}

One subtlety with propositionnal equality is dealing with dependant types: if $P$ is a proposition over a type $X$, and $x,y : X$, even if $(x = y)$ is inhabited, $(P(x) = P(y))$ does not make any sense : there is no reason for $P(x)$ and $P(y)$ to be of the same type! However, path induction allows us to construct a \emph{transport}:

\[\transp^{P} : \prod_{p : (x = y)} P(x) \to P(y) \]

Which reads: `for any path $p$ from $x$ to $y$, we can construct a function $p^{*}$ that maps from $P(x)$ to $P(y)$'. It follows that $P(x)$ and $P(y)$ are logically equivalent, and so equality behaves as expected. In fact, $\transp^{P}(p)$ is always an equivalence with inverse $\transp^{P}(p^{-1})$. Furthermore, one can show that equality for dependant pairs has a convenient shape:

\[(a, b) = (a', b') \simeq \sum_{p : a = a'}p^*(b) = b'\]

Which reads: `Two objects with structure are equal if and only if the objects are equal, and the transported structure of the first is equal to the structure of the second'. (Note that for non-dependant pairs one only needs to show that the two components are equal).

\section{An Example of Algebraic Structures in HoTT:\  $G$-Sets}

Before we dive in, a few notes on HoTT in Agda. The cubical library is the most commonly used implementation of HoTT in the Agda proof assistant (which is based on classical dependant type theory). It implements a specific branch called Cubical HoTT, which uses a formal interval $I$ which behaves like $[0;1]$, and is used to build paths. The cubical library has many results already formalized and checked in agda. I will mark those that I have checked in agda during my internship with the label (agd*), and those that I have proved on paper (pap*). All proofs are available in the appendix.


\subsection{Groups in HoTT}

Groups in HoTT are defined as a tuple (nested sum-type):

\begin{itemize}
\begin{minipage}[c][3.2cm]{0.3\linewidth}
    \item $G : \mathcal{U}$
    \item $1_{G} : G$
    \item $\mu : G \to G \to G$
    \item $\iota : G \to G$
\end{minipage}
\begin{minipage}[c][3.2cm]{0.6\linewidth}
    \item $G$ is a set ($\isset(G)$)
    \item $\mu$ is associative ($\prod_{x,y,z : G} \mu(x,(\mu(y,z))) = \mu(\mu(x,y),z)$)
    \item $1_{G}$ is a neutral element (both left and right)
    \item $\iota(g)$ is an inverse of $g$ (both left and right)
\end{minipage}
\end{itemize}

We will often note G both the carrier of the group (the underlying set) and the group itself, and $\_\cdot\_$ the group multiplication. We can define the type of all groups to be the sum-type of types in $\mathcal{U}$ along with a $\mu$ and $\iota$, that satisfy these conditions.

\subsection{Group Actions and G-Sets}

A Group Action of $G \curvearrowright X$ is a tuple:

\begin{itemize}
\begin{minipage}[c][3cm]{0.3\linewidth}
        \item $\mu : G \to X \to X$
\end{minipage}
\begin{minipage}[c][3cm]{0.6\linewidth}
        \item $\displaystyle\prod_{x : X} \mu(1_{G}, x) = x $
        \item $\displaystyle\prod_{g_1, g_2 : G}\prod_{x : X} \mu(g_1 \cdot g_2, x) = \mu(g_1, \mu(g_2,x)) $
        \item $\isset(X)$
\end{minipage}
\end{itemize}

And a $G$-Set is a carrier $X : \mathcal{U}$ and an action of $G$ on $X$ (called a `$G$-Set structure'). Note that $X$ must therefore be a set. A \emph{morphism} of $G$-Sets $X$ and $Y$ is a function between carriers that respects the actions:

\begin{gather*}
f : X \to Y \\
f(\mu_{X}(g,x)) = \mu_{Y}(g,f(x))
\end{gather*}

When $f$ is an equivalence, we call it an $G$-Set \emph{isomorphism}.

\subsection{From Isomorphisms to Paths}

We want to show that two isomorphic $G$-Sets can be identified. That is, we want an equivalence between the type $(X = Y)$, and the type of $G$-Set isomorphisms of $X$ to $Y$. Let us exhibit such an equivalence. This will come in handy when we get to constructing Eilenberg-MacLane spaces, and will help us get a feel for how univalence is used in practice.

\paragraph{Transporting structure:}

If $e$ is a bijection between carriers $X$ and $Y$, it is natural that $e$ induces a $G$-Set structure on $Y$. In HoTT this is immidiatley given by univalence because $e$ gives us a path that allows us to transport the $G$-Set structure of $X$ onto $Y$:

\[\transp^{\gsetstr}(\ua(e)) : \gsetstr(X) \to \gsetstr(Y)\]

We can actually calculate this specific transport (*agd).

Most notably, the transported action is:

\[g,y \mapsto e(\mu_{X}(g , e^{-1}(y)))\]

As for the proofs, they are all functions towards identity types on sets, thus functions towoards propositions. This means they are themselves proposition (this is simple to prove). Therefor there is only one choice for these proofs: the transported ones are equal to the ones from $Y$'s $G$-Set structure.

\paragraph{Building an equivalence:} Suppose we are given a path from $X$ to $Y$ as $G$-Sets. We have seen (\ref{transport}) that this type is equivalent to the type of pairs of a path between the carriers and a path between the transported structure of $X$ and the structure of $Y$. The first induces an equivalence $e$ (by $\pathtoequiv$) and the second is of type:

\[\prod_{g : G} \prod_{y : Y} e(\mu_{X}(g, e^{-1}(y))) = \mu_{Y}(g, y)\]

It is not too hard to see that this is equivalent to $e$ respecting the actions of $G$ (*agd). Therefor the type of paths from $X$ to $Y$ is equivalent to the type of equivalences between carriers that respect the actions, i.e.\ the type of isomorphisms.

\section{Truncation and Connectedness}

\subsection{$n$-truncated types}

When considering a given type, one might wonder what that type would look like if it was to be truncated to an $n$-type. That is, if we `forgot' all of the higher structure. For instance we want the $0$-truncation (or set-truncation) of $A$ (denoted $\|A\|_{0}$) to have `the same inhabitants' as $A$ with all paths contracted. Similarly, the propositional truncation of $A$ ($\|A\|_{-1}$) will be $\mathbb{1}$ if $A$ is inhabited and $\mathbb{0}$ if it is not. We define these objects with a universal property:

INSERT COMMUTING DIAGRAM

Where $X_{n}$ is an $n$-type. This reads `when mapping out of $\|A\|_{n}$ to an $n$-type, it suffices to map out of A. For example, when one wants to prove a proposition using an inhabitant of $\|A\|_{-1}$, one may assume he is given a proper inhabitant of $A$.

\subsection{Connected Types}

We've seen that a type $A$ is a proposition if all of its inhabitants are equal.

\[\isprop(A) :\equiv \prod_{x,y : A} (x = y)\]

But what if $(x = y)$ is merely inhabited ? We then say that $A$ is \emph{connected}:

\[\isconn(A) :\equiv \prod_{x,y : A} \|x = y \|_{-1}\]

Here the topological interpretation helps. The circle $\mathcal{S}^{1}$ is an example of a type that is connected but not a proposition.

\subsection{Connected Component}

We define the connected component of a pointed type $(A,a_0)$:

\[\baut(A,a_0) :\equiv \sum_{x : A} (x = a_0)\]

It is fairly intuitive that this type is connected (*agd), and that $\Omega(A,a_0) = \Omega(\baut(A,a_0))$. (The loop space of $a_0$ is its loop space in its connected component) (*agd). We are now ready to construct Eilenberg-MacLane spaces!

\section{The Torsor Construction}



\end{document}
